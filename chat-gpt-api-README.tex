\section{ChatGPT API
ハンズオン}\label{chatgpt-api-ux30cfux30f3ux30baux30aaux30f3}

ここでは、Deno で ChatGPT の API を利用する方法を紹介します。

\subsection{1. 🔧 準備編}\label{ux6e96ux5099ux7de8}

\subsubsection{1-1. 🧐 ChatGPT とは?}\label{chatgpt-ux3068ux306f}

OpenAI が開発した人工知能チャットボットです。

2023年7月現在、公式ページから、GPT 3.5 が無料で使用できます。\\
今回は OpenAI API を使用して ChatGPT を使用します。

\subsubsection{1-2. ✏️
OpenAIのアカウントを作成しよう}\label{openaiux306eux30a2ux30abux30a6ux30f3ux30c8ux3092ux4f5cux6210ux3057ux3088ux3046}

API や ChatGPT の利用にはアカウント登録が必要になります。\\
登録済みの人はそのアカウントを使用してください。

まず、\href{https://openai.com}{OpenAI} に移動します。

右上から、Sign Up を選択して、新規登録画面へ移動しましょう。

ここからは、各自でアカウントの登録をお願いします🙏\\
完了時に以下のような画面になっていればOKです。

\textbf{登録はここまでで完了です}

\begin{center}\rule{0.5\linewidth}{0.5pt}\end{center}

新規登録・ログインができたら、完了時に表示されていた画面で、API
を選択します。

\subsubsection{1-3. ⚙️ クレジットと API
キーの確認}\label{ux30afux30ecux30b8ux30c3ux30c8ux3068-api-ux30adux30fcux306eux78baux8a8d}

続いて、クレジットとAPI キーを確認していきます。

自分のプログラムから、API を呼び出すには API キーが必要になります。\\
また、クレジットを消費します。\\
クレジットがない場合は課金する必要があります。

画面右上の自分のアイコンと Personal
と書いてある部分をクリック、出てきたメニューから Manage Account
を選択します。

画面が移動したら、左側メニューから Usage を選択してください。

Usage では自分がどれくらい API を使用したかが確認できます。\\
無料で使用できるクレジットが Free trial usage
というセクションから確認できます。

この図では \$5.00 が無料枠として付与されています。

この部分では、Used が使用済みクレジット、 Expired
が期限切れのクレジットを示しています。

つまり、この図の状態だと無料枠が残っていないので、使用したい場合は課金する必要があるということになります。

\begin{center}\rule{0.5\linewidth}{0.5pt}\end{center}

続いて、API キーの確認をしていきます。\\
今度は、左のメニューから、API Keysを選択します。

ここでは、発行したAPI キーのリストを確認することができます。 Secret Key
という名前のキーがデフォルトで発行されているはずです。

ここでは、インターン用に新しく1つキーを発行していきます。\\
「+ Create new secret key」を選択します。

名前は各自におまかせしますが、「jigintern」のように何に使っているかわかりやすい命名にしておくのをオススメします。

名前を入力したら、Create secret key を押します。

下の図でyou won't be able to view it
againと書いてある通り、\textbf{ここで出てきたAPIキーは Done
を選択すると、この後は見れなくなります。}

なので、発行したキーはどこかに保管しておいてください。

ちなみに、万が一わからなくなった場合は、そのキーを削除して、新しくキーを作ってしまえばOKです。

2個以上ある場合は、ゴミ箱アイコンを押せば消せます。

\subsection{2. 💪 実践編}\label{ux5b9fux8df5ux7de8}

それでは、準備はできたので、実際に API を使用してみます。

とはいえ、実際の API
仕様を確認してイチから実装するのも厳しいので、\href{https://github.com/code4fukui/ai_chat}{こちら}のfetchChat
/ fetchConversation を使用します。

\subsubsection{2-1. 😎
サンプルを元に動かすところまでやってみよう}\label{ux30b5ux30f3ux30d7ux30ebux3092ux5143ux306bux52d5ux304bux3059ux3068ux3053ux308dux307eux3067ux3084ux3063ux3066ux307fux3088ux3046}

ということで、今開いている README.md
と同じ階層にある、\texttt{chatGPT-api-handson/sample}
フォルダに移動します。

移動先のフォルダには Deno と \texttt{fetchChat} /
\texttt{fetchConversation} を利用したサンプルコードが入っています。

\texttt{fetchChat} / \texttt{fetchConversation} も内部実装では OpenAI
API を使用しているので、API キーが必要になります。

\texttt{.env}ファイルを用意して、そこにAPIキーを書き込んでおくことで自動でAPIキーを読み込んでくれます。\\
今回はあらかじめ雛形を
\texttt{.env.example}という名前で用意したので、\texttt{.env}
に改名して使用します。\\
(つまり、ファイル名から \texttt{.example} を消してください)

改名したら、ファイルの中身を書き換えていきます。\\
\texttt{OPENAI\_API\_KEY=} までを残して \texttt{sk-}
以下を削除します。\\
削除したら、\texttt{=} の後に、APIキーを貼り付けましょう。

これで下準備はOK!\\
コードの中身は、この後解説するので、ひとまずChatGPTが動くか試してみましょう!

実行には Deno の実行環境が必要ですが、以前の内容で Deno
の環境構築は終わっていると思います。\\
下記のとおりコマンドを実行してください。

\begin{Shaded}
\begin{Highlighting}[]
\ExtensionTok{./serve.ts}
\end{Highlighting}
\end{Shaded}

うまく動かない人は、下記でも大丈夫です。

\begin{Shaded}
\begin{Highlighting}[]
\ExtensionTok{deno}\NormalTok{ run }\AttributeTok{{-}{-}allow{-}read} \AttributeTok{{-}{-}allow{-}net} \AttributeTok{{-}{-}allow{-}write} \AttributeTok{{-}{-}allow{-}env} \AttributeTok{{-}{-}watch}\NormalTok{ serve.ts}
\end{Highlighting}
\end{Shaded}

Deno Deploy を使用する際は、APIキーは Deno Deploy
側で設定する必要があります。\\
各チームで担当のメンターに対応してもらってください。

\subsubsection{2-2. 👀
実装をのぞいてみよう}\label{ux5b9fux88c5ux3092ux306eux305eux3044ux3066ux307fux3088ux3046}

ChatGPTが動くことを確認できたでしょうか。\\
今度はプログラムがどのようになっているのかを確認していきます。

OpenAI API を呼び出して、返答を受け取るまでの処理に着目します。\\
おおまかには index.html / chat.html (フロントエンド) -\textgreater{}
serve.ts (バックエンド) -\textgreater{} OpenAI API
という順番で呼び出しています。

フロントエンド側は、index.html から fetch API
を使用して以下のように呼び出しています。

\begin{Shaded}
\begin{Highlighting}[]
\KeywordTok{const}\NormalTok{ response }\OperatorTok{=} \ControlFlowTok{await} \FunctionTok{fetch}\NormalTok{(}\StringTok{\textquotesingle{}/api/conversation\textquotesingle{}}\OperatorTok{,}\NormalTok{ \{}
  \DataTypeTok{method}\OperatorTok{:} \StringTok{\textquotesingle{}POST\textquotesingle{}}\OperatorTok{,}
  \DataTypeTok{headers}\OperatorTok{:}\NormalTok{ \{}
    \StringTok{\textquotesingle{}Content{-}Type\textquotesingle{}}\OperatorTok{:} \StringTok{\textquotesingle{}application/json\textquotesingle{}}
\NormalTok{  \}}\OperatorTok{,}
  \DataTypeTok{body}\OperatorTok{:} \BuiltInTok{JSON}\OperatorTok{.}\FunctionTok{stringify}\NormalTok{(\{ }\DataTypeTok{messages}\OperatorTok{:}\NormalTok{ messages \})}
\NormalTok{\})}\OperatorTok{;}

\ControlFlowTok{if}\NormalTok{ (}\OperatorTok{!}\NormalTok{response}\OperatorTok{.}\AttributeTok{ok}\NormalTok{) \{}
  \ControlFlowTok{return} \StringTok{\textquotesingle{}Error: \textquotesingle{}} \OperatorTok{+}\NormalTok{ response}\OperatorTok{.}\AttributeTok{status}\OperatorTok{;}
\NormalTok{\}}

\KeywordTok{const}\NormalTok{ data }\OperatorTok{=} \ControlFlowTok{await}\NormalTok{ response}\OperatorTok{.}\FunctionTok{json}\NormalTok{()}\OperatorTok{;}
\ControlFlowTok{return}\NormalTok{ data}\OperatorTok{.}\AttributeTok{message}\OperatorTok{;}
\end{Highlighting}
\end{Shaded}

\texttt{/api/conversation} に対して、POSTリクエストを送っていますね。

では、バックエンド側はどのように実装しているのか確認してみましょう!

\begin{Shaded}
\begin{Highlighting}[]
\KeywordTok{const}\NormalTok{ json }\OperatorTok{=} \ControlFlowTok{await}\NormalTok{ request}\OperatorTok{.}\FunctionTok{json}\NormalTok{()}\OperatorTok{;}
\BuiltInTok{console}\OperatorTok{.}\FunctionTok{log}\NormalTok{(json)}\OperatorTok{;}
\KeywordTok{const}\NormalTok{ response }\OperatorTok{=} \ControlFlowTok{await} \FunctionTok{fetchConversation}\NormalTok{(json}\OperatorTok{.}\AttributeTok{messages}\OperatorTok{,} \KeywordTok{null}\OperatorTok{,} \KeywordTok{true}\NormalTok{)}\OperatorTok{;}
\BuiltInTok{console}\OperatorTok{.}\FunctionTok{log}\NormalTok{(response)}\OperatorTok{;}
\ControlFlowTok{return} \KeywordTok{new} \FunctionTok{Response}\NormalTok{(}\BuiltInTok{JSON}\OperatorTok{.}\FunctionTok{stringify}\NormalTok{(\{ }\DataTypeTok{message}\OperatorTok{:}\NormalTok{ response \}))}\OperatorTok{;}
\end{Highlighting}
\end{Shaded}

まず、リクエストをjson形式で受け取ります。
フロントエンド側からメッセージが、messagesという名前で送られてきているので、バックエンド側では\texttt{json.messages}
というようにして利用しています。\\
\texttt{fetchConversation} で、OpenAI API
を呼び出して、ChatGPTからの返答を受け取っています。\\
chat.html から呼び出している fetchChat も同様に実装しています。

このように、serve.ts で OpenAI API を呼び出す API を作成し、html
側からは作成した API を呼び出すことで、ChatGPT
の返答をhtml(フロントエンド)側まで持ってくることができます。

最後に fetchChat / fetchConversation の呼び出し方・使い方を学習します。

\subsubsection{\texorpdfstring{2-3. 😀 \texttt{fetchChat} /
\texttt{fetchConversation}
の使い方}{2-3. 😀 fetchChat / fetchConversation の使い方}}\label{fetchchat-fetchconversation-ux306eux4f7fux3044ux65b9}

\paragraph{\texorpdfstring{\texttt{fetchChat}}{fetchChat}}\label{fetchchat}

\texttt{fetchChat} は引数として1メッセージのみを送り、返り値として
ChatGPT からの返答メッセージを受け取ります。

サンプルの実装では下記のようになっています。

\begin{Shaded}
\begin{Highlighting}[]
\KeywordTok{const}\NormalTok{ response }\OperatorTok{=} \ControlFlowTok{await} \FunctionTok{fetchChat}\NormalTok{(json}\OperatorTok{.}\AttributeTok{message}\NormalTok{)}\OperatorTok{;}
\end{Highlighting}
\end{Shaded}

引数として、ユーザーの入力したメッセージを文字列型(string)で指定するだけでOKです。\\
受け取ったレスポンス response も文字列型になっています。

\paragraph{\texorpdfstring{\texttt{fetchConversation}}{fetchConversation}}\label{fetchconversation}

\texttt{fetchConversation} は
ユーザーとChatGPTの複数のメッセージを会話として送ることで、返り値としてChatGPTからの返答メッセージを受け取ります。

そのため、\texttt{fetchChat}
では入力1つに対して出力1つでしたが、会話そのもののやりとりから返答を考えてもらうことができます。

サンプルの実装では下記のようになっています。

\begin{Shaded}
\begin{Highlighting}[]
\KeywordTok{const}\NormalTok{ response }\OperatorTok{=} \ControlFlowTok{await} \FunctionTok{fetchConversation}\NormalTok{(json}\OperatorTok{.}\AttributeTok{messages}\OperatorTok{,} \KeywordTok{null}\OperatorTok{,} \KeywordTok{true}\NormalTok{)}\OperatorTok{;}
\end{Highlighting}
\end{Shaded}

引数が3つ指定されていますが、1つ目がやり取りのメッセージ、2つ目が関数呼び出し用の関数、3つ目が
GPT3.5 を使用するかどうかのbool値になっています。

1つ目の引数のメッセージは以下のような形式 (Object) になっています。

role は、\texttt{user} がユーザー、 \texttt{assistant} がChatGPT
からの返答となっています。

\begin{Shaded}
\begin{Highlighting}[]
\NormalTok{\{}
  \DataTypeTok{messages}\OperatorTok{:}\NormalTok{ [}
\NormalTok{    \{ }\DataTypeTok{role}\OperatorTok{:} \StringTok{"user"}\OperatorTok{,} \DataTypeTok{content}\OperatorTok{:} \StringTok{"こんにちは"}\NormalTok{ \}}\OperatorTok{,}
\NormalTok{    \{ }\DataTypeTok{role}\OperatorTok{:} \StringTok{"assistant"}\OperatorTok{,} \DataTypeTok{content}\OperatorTok{:} \StringTok{"こんにちは!ご用件は何でしょうか?"}\NormalTok{ \}}\OperatorTok{,}
\NormalTok{    \{ }\DataTypeTok{role}\OperatorTok{:} \StringTok{"user"}\OperatorTok{,} \DataTypeTok{content}\OperatorTok{:} \StringTok{"カレーの作り方を教えてください。"}\NormalTok{ \}}
\NormalTok{  ]}
\NormalTok{\}}
\end{Highlighting}
\end{Shaded}

2つ目の引数の、関数呼び出しは使わない場合は null で OK
です。今回は特に触れません。

3つ目の引数のGPT-3.5を使用するかどうかは、GPT-4 を使用する場合は false
にする必要があります。\\
今回は GPT-3.5 で十分かなと思いますので、true のままで大丈夫です。
